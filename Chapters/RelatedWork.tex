\chapter{Related Work and Background} % Main chapter title

\label{RelatedWork} % For referencing the chapter elsewhere, use \ref{Chapter1} 

\lhead{Chapter 2. \emph{Related Work and Background}} % This is for the header on each page - perhaps a shortened title

%----------------------------------------------------------------------------------------
%	SECTION 1
%----------------------------------------------------------------------------------------

Intro

%----------------------------------------------------------------------------------------
%	SECTION 1
%----------------------------------------------------------------------------------------

\section{Row-oriented Storage and Data Access}
\label{sec:row-oriented}

%----------------------------------------------------------------------------------------
%	SECTION 1
%----------------------------------------------------------------------------------------

\section{Column-stores}
\label{sec:column-stores}

%----------------------------------------------------------------------------------------
%	SECTION 1
%----------------------------------------------------------------------------------------

\section{Indices}
\label{sec:indices}

%----------------------------------------------------------------------------------------
%	SECTION 1
%----------------------------------------------------------------------------------------

\section{Offline Indexing}
\label{sec:holistic}

Offline indexing is the earliest approach on self-tuning database systems.
Nowadays, all major database products offer auto-tuning tools \cite{DBLP:conf/vldb/AgrawalCKMNS04,DBLP:conf/vldb/DagevilleDDYZZ04, DBLP:conf/vldb/ZilioRLLSGF04} to automate the database physical design.
Auto-tuning tools mainly rely on a ``what-if analysis'' \cite{DBLP:conf/sigmod/ChaudhuriN98} and close interaction with the optimizer \cite{DBLP:conf/vldb/ChaudhuriN97} to decide which indices are potentially more useful for a given workload.

Offline indexing  requires heavy involvement of a database administrator (DBA).
Specifically, a DBA invokes the tool and provides its input, i.e., a representative workload.
The tool analyzes the given workload and recommends an appropriate physical design.
However, the DBA is the one that decides which of the changes in the physical design should be applied.
The main limitation of offline indexing appears when the workload cannot be predicted and/or there is not enough idle time to invest in the offline analysis and the physical design implementation. 


%----------------------------------------------------------------------------------------
%	SECTION 1
%----------------------------------------------------------------------------------------

\section{Online Indexing}
\label{sec:online}

Online indexing addresses the limitation of offline indexing.
Instead of making all decisions a priori, the system continuously monitors the workload and the physical design is periodically reevaluated.
System COLT \cite{DBLP:conf/sigmod/SchnaitterAMP06} was one of the first online indexing approaches.
COLT continuously monitors the workload and periodically in specific epochs, i.e., every N queries, it reconsiders the physical design.
The recommended physical design might demand creation of new indices or dropping of old ones.
COLT requires many calls to the optimizer to obtain cost estimations.
A ``lighter'' approach, i.e., requiring less calls to the optimizer, was proposed later \cite{DBLP:conf/icde/BrunoC07}.
Soft indices \cite{DBLP:conf/icde/LuhringSSS07} extended the previous 
online approaches by building full indices on-the-fly concurrently with queries on the same data, sharing the scan operator.

The main limitation of online indexing is that reorganization  of the physical design can be a costly action
that a) requires a significant amount of time to complete and b) requires a lot of resources.
This means that online indexing is appropriate mainly for moderately dynamic workloads where the query patterns do not change
very frequently. Otherwise, it may be that by the time we finish adapting the physical design, the workload has changed again
leading to a suboptimal performance.


%----------------------------------------------------------------------------------------
%	SECTION 1
%----------------------------------------------------------------------------------------

\section{Adaptive Indexing}
\label{sec:cracking}


Adaptive indexing is the latest and the most lightweight approach in self-tuning databases.
Adaptive indexing  addresses the limitations of offline and online indexing for dynamic workloads; 
it  instantly adjusts to workload changes by building or refining indices partially and incrementally as part of query processing.
By reacting to every single query with lightweight actions, adaptive indexing manages to instantly adapt
to a changing workload. As more queries arrive, the more the indices are refined 
%according to the workload 
and the more performance improves.
Adaptive indexing has been studied in the context of main-memory column-stores \cite{DBLP:conf/cidr/IdreosKM07, CrackRepeat}, Hadoop \cite{Hail}
as well as for improving more traditional disk-based settings \cite{DBLP:conf/edbt/GraefeK10}.
It has been shown to work for many core database architecture issues 
such as updates \cite{DBLP:conf/sigmod/IdreosKM07}, multi-attribute queries \cite{DBLP:conf/sigmod/IdreosKM09}, concurrency control \cite{multicore_adaptive, DBLP:journals/pvldb/GraefeHIKM12, DBLP:journals/vldb/GraefeHIKMS14}, partition-merge-like logic \cite{DBLP:conf/edbt/GraefeK10,DBLP:journals/pvldb/IdreosMKG11}. In addition, \cite{DBLP:conf/tpctc/GraefeIKM10} shows how to benchmark adaptive indexing techniques, while stochastic database cracking \cite{DBLP:journals/pvldb/HalimIKY12} shows how to be robust on various workloads and \cite{IndexingKeys} shows how adaptive indexing can apply to key columns.
Finally, recent work on parallel adaptive indexing studies CPU-efficient implementations and proposes algorithms to exploit multi-cores \cite{multicore_adaptive,efficient_cracking}.
 
The main limitation of adaptive indexing is that it works only during query processing.
In this way, the only opportunity to improve the physical design is only when queries arrive.

Recently, adaptive indexing concepts have been extended to provide adaptive indexes for time series data \cite{DBLP:conf/sigmod/ZoumpatianosIP14}
as well as using incoming queries for more broad storage layout decisions, 
i.e., reorganizing base data (columns/rows) according to incoming query requests \cite{Alagiannis:2014:HHA:2588555.2610502}, 
or even about which data should be loaded \cite{DBLP:conf/cidr/IdreosAJA11}.
In addition, adaptive indexing ideas have been used to design new generation data exploration tools such as 
touch-based data systems \cite{DBLP:conf/cidr/IdreosL13, DBLP:conf/icde/LiarouI14}.

%----------------------------------------------------------------------------------------
%	SECTION 1
%----------------------------------------------------------------------------------------

\section{Database Systems for the Multi-core Era}
\label{sec:multicore_systems}

Modern hardware offers opportunities for high parallelism;
%even
a single machine may be equipped with chip multiprocessors, which contain multiple cores with support for multiple context threads.
Recent research focuses on exploiting parallelism opportunities by a) processing multiple queries concurrently, 
and b) by 
%trying to 
parallelizing 
%every 
tasks in the critical path during query processing
\cite{mc1,DBLP:conf/cidr/HarizopoulosA03,DBLP:conf/sigmod/HarizopoulosSA05,mc2}.
Sorting is one of the most important database tasks (and a core component of adaptive indexing in column-stores) that can be highly-parallelized using modern hardware advances \cite{hash_join_rev, hash_sort_SIMD, Polychroniou:2014:CSM:2588555.2610522, database_SIMD}.


%----------------------------------------------------------------------------------------
%	SECTION 1
%----------------------------------------------------------------------------------------

\section{The MonetDB System}
\label{sec:monetdb}


%----------------------------------------------------------------------------------------
%	SECTION 1
%----------------------------------------------------------------------------------------

\section{Summary}
\label{sec:summaryRelated}

