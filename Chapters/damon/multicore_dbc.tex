\chapter{Database Cracking: Fancy Scan, not Poor Man's Sort!}

\label{damon}

\lhead{Chapter 3. \emph{Database Cracking: Fancy Scan, not Poor Man's Sort!}} % This is for the header on each page - perhaps a shortened title


%----------------------------------------------------------------------------------------
%	SECTION 1: ABSTRACT
%----------------------------------------------------------------------------------------

	
  Database Cracking is an appealingly simple approach to adaptive
  indexing: on every range-selection query, the data is partitioned
  using the supplied predicates as pivots. The core of database
  cracking is, thus, pivoted partitioning. While pivoted partitioning,
  like scanning, requires a single pass through the data it tends to
  have much higher costs due to lower CPU efficiency. In this paper,
  we conduct an in-depth study of the reasons for the low CPU
  efficiency of pivoted partitioning. Based on the findings, we
  develop an optimized version with significantly higher
  (single-threaded) CPU efficiency. We also develop a number of
  multi-threaded implementations that are effectively bound by memory
  bandwidth. Combining all of these optimizations we achieve an
  implementation that has costs close to or better than an ordinary
  scan on a variety of systems ranging from low-end (cheaper than
  \$300) desktop machines to high-end (above \$10,000) servers.


