\section{Introduction}
\label{sec:intro}

\begin{table*}
\begin{center}

\begin{tabular}{ccccccc}
\hline
\cline{1-7}
Indexing& Statistical&Exploitation of idle resources & Exploitation of idle resources & Index  		 & Updates, & Workload\\
        & analysis   & \emph{before} query processing& \emph{during} query processing & materialization  & projection cost &\\
\hline
Offline & $\surd$ & $\surd$ & $\times$  & full & high & static \\ 
Online & $\surd$ & $\times$ & $\surd$  & full & high & dynamic\\
Adaptive & $\times$ & $\times$ & $\times$  & partial & low & dynamic\\
Holistic & $\surd$ & $\surd$ & $\surd$  & partial & low & dynamic\\
\hline
\label{tab:qualitative}
\end{tabular}
\vspace{-0.45cm}
 \caption{Qualitative difference among offline, online, adaptive and holistic indexing.}
\vspace{-0.7cm}
\end{center}
\end{table*}

The big data era is causing the research community to rethink fundamental issues 
in the design of database systems  
towards more usable systems \cite{UsableDatabases} that can access data better and faster 
%\cite{blinkdb,MAD,shark,exploration,hinge, hazy,earl,GuidedInteraction, RequirementsSciDB, datacuration,energypartitioning},
\cite{exploration,hinge, hazy,earl},
 that can better exploit modern hardware and opportunities for massive parallelization
%\cite{swissbox,elastras,hadoop+,cod,dbseer},
\cite{hadoop+}
 that can support efficient processing of OLTP and/or OLAP queries
% \cite{nodb,shareddb,hstore,hyper,bitweaving}
 \cite{nodb,hstore,hyper,bitweaving}.
% or even exploit crowdsourcing for query processing \cite{crowdq,crowddb}.

\textbf{The Physical Design Problem.} 
Physical design, and in particular proper index selection, is a predominant factor for the performance of database systems and has only become more crucial in the big data era.
With new dynamic and exploratory environments, physical design becomes especially hard given the instability of workloads and the continuous stream of big data; a single physical design choice is not necessarily correct or useful for long stretches of time, while at the same time workload knowledge is scarce given the exploratory user behavior.

\textbf{State-of-the-Art.} In database applications, where ``the future is known'', physical design is assigned to database administrators
who may also be assisted by auto-tuning tools 
\cite{DBLP:conf/vldb/AgrawalCKMNS04,DBLP:conf/vldb/DagevilleDDYZZ04,DBLP:conf/vldb/ZilioRLLSGF04}.
Still though, a significant amount of human intervention is necessary and everything needs to happen offline.
Thus, offline indexing can be applied with good results only on applications where there is enough workload 
knowledge and idle time to prepare the physical design appropriately before queries arrive.

Unfortunately, for many modern applications ``the future is unknown'', e.g., in scientific databases, social networks, web logs, etc.
In particular, the query processing patterns follow an exploratory behavior, which changes so arbitrarily that it cannot be predicted.
Such environments cannot be handled by offline indexing.
Online indexing \cite{DBLP:conf/icde/BrunoC07,DBLP:conf/sigmod/SchnaitterAMP06} 
and adaptive indexing \cite{DBLP:conf/cidr/IdreosKM07} 
are two approaches to automatic physical design in such dynamic environments, but none of them in isolation handles the problem sufficiently.
Online indexing periodically refines the physical design but it may negatively affect running queries every time
it needs to use resources for reconsidering the physical design and it may not be quick to follow the workload changes as it reacts only periodically. 
Adaptive indexing does not have this problem as it introduces continuous, incremental and partial index refinement 
but it adjusts the physical design only during query processing based on queries.

\textbf{Always Indexing.}
In this paper, we make the observation that in real systems there are plenty of resources that remain under-utilized and 
we propose to exploit those resources to be able to better address dynamic and  ad-hoc environments. 
In particular, we focus on exploiting CPU cycles to the maximum by continuously detecting idle CPU cycles
and using them to refine the physical design (in parallel with query processing).
Such idle CPU cycles occur when the system  does not exploit all available cores up to 100\%.
% (non-peak workload).
%i.e., either because the workload is not enough to saturate the CPUs or because the current tasks performed for 
%query processing are not easy to parallelize to the point where all available CPU power can be  exploited
%or when there are no active user queries (\emph{idle time}).
We distinguish between ``\emph{idle time}'' as in
``\emph{there is no user-driven workload at all and the entire CPU (all its hardware contexts) is idle (except possible occasional operating system background activity)}''
and ``\emph{idle CPU resources}'' as in
``\emph{the active user-driven workload does / can not use all physically available CPU hardware contexts.}'' 
Intuitively, there are two options when resources are under-utilized but still there are active queries in the system.
The first option is to introduce more parallel query processing algorithms to maximize utilization for the existing workload. The second one is to exploit the extra resources towards a different goal (extra indexing actions in our case). We investigate and compare both directions.


%\begin{figure}[t]
%\begin{center}
%\includegraphics[trim=4cm 22cm 12cm 2.8cm]{qualitative}
%\caption{Indexing Approaches.}
%\vspace{-0.4 in}
%\label{fig:various}
%\end{center}
%\end{figure}

\textbf{Holistic Indexing.} In this paper, we introduce a new indexing approach, 
called \emph{holistic indexing}.
Holistic indexing  addresses the automatic physical design problem in 
modern applications with  dynamic and exploratory workloads.
%\textcolor{red}{where there is no prior knowledge about how many under-utilized CPU resources will be available}.  
It continuously monitors the workload and the CPU utilization; 
when idle CPU cycles are detected, holistic indexing exploits them in order to partially and incrementally 
adjust the physical design based on the collected statistical information.
Each index refinement step may take only a few microseconds to complete and the system will typically perform
several such steps in one go depending on available system resources.
Everything happens in parallel to query processing but without disturbing running queries.
The net effect is that holistic indexing refines the physical design,
% there are more chances for future queries to enjoy better  data access patterns.
improving performance and robustness by enabling better data access patterns for future queries.
Table~\ref{tab:qualitative} summarizes the qualitative difference between holistic indexing and current state-of-the-art indexing approaches.
Compared to past approaches, holistic indexing manages to minimize both initialization and maintenance costs, 
as it relies on partial indexing, and to exploit all possible CPU resources
%before and/or during query processing, 
in order to provide a more complete physical design.

\textbf{Contributions.}
Our contributions are as follows:
\vspace{-0.5em}
\begin{itemize}
  \setlength{\itemsep}{0pt}
  \setlength{\parskip}{0pt}
  \setlength{\parsep}{0pt}
\item We introduce the idea of exploiting idle CPU resources towards 
continuously adapting the physical design to ad-hoc and dynamic workloads.
\item We discuss  in detail the design of holistic indexing on top of modern column-store architectures, i.e.,
how to detect and exploit idle CPU resources during query processing.
\item We implemented holistic indexing in  an open-source column-store, MonetDB \cite{DBLP:journals/debu/IdreosGNMMK12, monetdb}.
%\textcolor{red}{The code is publicly available in \cite{holindex}.} 
Through a detailed experimental analysis both with microbenchmarks and with TPC-H 
we demonstrate that we can exploit idle CPU resources to prepare the physical design better,
leading to significant improvements over past indexing approaches in dynamic environments. 
\end{itemize}
\vspace{-0.5em}

\textbf{Paper Structure.} The rest of the paper is structured as follows:
Section~\ref{sec:relwork} provides an overview of related work.
In Section~\ref{sec:background}, we shortly recap the basics of column-store architectures 
and the basics of adaptive indexing.
Then, Section~\ref{sec:problem} introduces holistic indexing, while Section~\ref{sec:experiments} presents a detailed experimental analysis.
Finally, Section~\ref{sec:conclusions} discusses future work and concludes the paper. 


